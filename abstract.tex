\thispagestyle{plain}% This way, we can have the page numbering in the foot.
\selectlanguage{english}
\addcontentsline{toc}{chapter}{\numberline{}Abstract}
\begin{abstract}
	\begin{verse}
	\small \textit{
	by Quentin Delhaye,  Master in Computer Science and Engineering, Professional Focus in Software and Critical Systems Design, Université Libre de Bruxelles, 2014--2015.}
	\normalsize
	\end{verse}
	\begin{center}
		\textbf{Implementation of high level cryptographic protocols using a SoC platform}
	\end{center}
The aim of this thesis is to study the impact on the performance and ressource usage of the implementation of cryptographic protocols on a SoC platform.
The platform consists of the combination of an ARM core with an Altera FPGA, on which Barco Silex' crypto IP cores are programmed.

The focus is set on specific cryptographic schemes: AES and its CBC mode of operation, as well as RSA and Diffie-Hellman, all being supported in hardware.
The GCM mode of AES is also part of the results of the software implementation. Doing so opens the discussion on the performance gain if the device driver could use the hardware.
The two main network protocols implemented to use those operations are TLS/SSL and IPsec.
Several applications are considered for the implementation, among which OpenVPN, OpenSSL, OpenSSH and Strongswan.
Implementing those on the platform allows realistic benchmarks on TLS connections, latency and file transfer.

The results are unequivocal: the overhead added by OpenVPN not only delivers poor performance (56\% drop without encryption nor authentication), but also do not benefit from the hardware, yielding even worse results when offloading.
With IPsec, on the other hand, the hardware uses four times less CPU than the software for similar performance on file transfer.
As for the TLS connections, the hardware is able to process up to 7.8 times more connections than the software at high security parameters, and for 19 times less CPU utilization.


% Network security, SSL/TLS and IPsec compared, different ways to offload cryptographic operations from the user and kernel space of the operating system.
~\newline{}
\textbf{Keywords:} \textit{Cryptography, network security, hardware offloading, SoC platform, IPsec}
\end{abstract}
\clearpage

\thispagestyle{plain}
\selectlanguage{francais}
\addcontentsline{toc}{chapter}{\numberline{}Résumé}
\begin{abstract}
	\begin{verse}
	\small \textit{
	par Quentin Delhaye, Master en ingénieur civil en informatique, à finalité Software and Critical Systems Design, Université Libre de Bruxelles, 2014--2015.}
	\normalsize
	\end{verse}
	\begin{center}
		\textbf{Implementation of high level cryptographic protocols using a SoC platform}
	\end{center}
Le but de ce mémoire est d'étudier l'implémentation de protocoles cryptographiques sur une plateforme SoC et leur impact sur les performances et l'utilisation des ressources.
La plateforme consiste en la combinaison d'un cœur ARM avec un FPGA d'Altera sur lequel sont programmées des IP cores de Barco Silex.

L'accent est mis sur certains algorithmes cryptographiques~: AES et son mode CBC, ainsi que RSA et Diffie-Hellman, tous étant supportés par le hardware.
Le mode GCM d'AES fait aussi partie des résultats des implémentations software, ouvrant ce faisant la discussion sur le gain de performance qu'apporterait un pilote tirant parti du hardware.
Les deux principaux protocoles réseaux utilisant ces opérations qui sont implémentés sont TLS/SSL et IPsec.
Plusieurs application sont utilisées pour implémenter ces protocoles~: OpenVPN, OpenSSH, OpenSSL et Strongswan.
Leur implémentation permet d'effectuer des tests réalistes de connexions TLS, de latence et de transfert de fichier.

Les résultats sont sans appel~: le coût imposé par OpenVPN non seulement diminue les performance (56~\% de perte sans chiffrement ni authentification), mais en plus il n'exploite pas le hardware, donnant des résultats encore plus mauvais quand les opérations sont déchargées sur le périphérique.
En revanche, IPsec combiné au hardware utilise quatre fois moins de CPU que le software pour des performances similaires de transfert de fichier.
Concernant les connections TLS, le hardware est capable de traiter jusqu'à 5.89 fois plus de connexions par minute que le software avec des paramètre de sécurité élevés, et ce en utilisant 19 fois moins de CPU.

~\newline{}
\textbf{Mots-clés:} \textit{Cryptographie, sécurité réseau, hardware offloading, plaeforme SoC, IPsec}
\end{abstract}