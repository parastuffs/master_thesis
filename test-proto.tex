\chapter{Test protocol and Prediction}

\section{Experimental setup}
The experimental environment is built around a standard x86 host and an ARM Cortex-A9 alongside an Altera Cyclone V FPGA as the target.
Both are linked toghether through a network capped by 100Mbps switches.
Both stations have gigabits ethernet interface and could hence be diretly connected to each other, but in that case the communication would be limited by the I/O transfers of the storage units -- a hard drive disk in one case, an micro-SD card in the second -- on which we can not depend to set a constant throughput limitation, as it is highly influenced by the data block size and general health of the support.


\subsection{x86 host}
The desktop host runs on Windows 7 Professional 64-bit, but a virtual machine using a Linux distribution is used for the developpement and testings.

\begin{framed}
\begin{description}
	\item[OS] Ubuntu 12.04 LTS, kernel 3.16
	\item[CPU] Intel Core-i3 ... (two logical core out of four)
	\item[RAM] 1GB DDR3
\end{description}
\end{framed}

\subsection{Altera Socrates SoCFPGA}

\begin{framed}
\begin{description}
	\item[OS] Yocto project, kernel 3.14
	\item[CPU] Dual core ARM Cortex-A9, 800MHz
	\item[RAM] ...GB DDR3
	\item[FPGA] Altera Cyclone V
\end{description}
\end{framed}

\subsection{ARM DS-5 Streamline}
% Use a daemon and a driver to get system info and send them to another machine. Has almost no impact on the performance and the load of the system.

\section{File transfer}
The file transfered is an un compressed block of 128MB of random data generated using the following command:
\begin{lstlisting}[language=bash]
  $ head -c $((1024*1024*128)) /dev/urandom > heavy.file
\end{lstlisting}
