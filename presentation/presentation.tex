\documentclass[xcolor={x11names, rgb, usenames, dvipsnames}]{beamer}
% Beamer loads xcolor by default. Do not load it a second time using \usepackage

\usepackage[francais, english]{babel}
\usepackage[T1]{fontenc}
\usepackage[utf8]{inputenc}
\usepackage{pgfplots}
\pgfplotsset{compat=newest}
\usepackage{graphicx}
\usepackage{hyperref}
\usepackage{amsmath}


\setbeamertemplate{bibliography item}{[\theenumiv]}

%\usetheme{Warsaw}
% \usetheme{Boadilla}
% \usetheme{Antibes}
\usetheme{CambridgeUS}
\usecolortheme{dolphin}
% \usetheme{Berlin}
% \usetheme{Madrid}
% \setbeamertemplate{footline}[frame number]



% http://tex.stackexchange.com/questions/116077/presentation-beamer-title-page
\makeatletter
\newcommand\titlegraphicii[1]{\def\inserttitlegraphicii{#1}}
\titlegraphicii{}
\setbeamertemplate{title page}
{
  \vbox{}
  \vspace{-2.5em}
   {\usebeamercolor[fg]{titlegraphic}\inserttitlegraphic\hfill\inserttitlegraphicii\par}
  \vskip2.5em
  \begin{centering}
    \begin{beamercolorbox}[sep=8pt,center]{institute}
      \usebeamerfont{institute}\insertinstitute
    \end{beamercolorbox}
    \begin{beamercolorbox}[sep=8pt,center]{title}
      \usebeamerfont{title}\inserttitle\par%
      \ifx\insertsubtitle\@empty%
      \else%
        \vskip0.25em%
        {\usebeamerfont{subtitle}\usebeamercolor[fg]{subtitle}\insertsubtitle\par}%
      \fi%     
    \end{beamercolorbox}%
    \vskip1em\par
    \begin{beamercolorbox}[sep=8pt,center]{date}
      \usebeamerfont{date}\insertdate
    \end{beamercolorbox}%\vskip0.5em
    \begin{beamercolorbox}[sep=8pt,center]{author}
      \usebeamerfont{author}\insertauthor
    \end{beamercolorbox}
  \end{centering}
  %\vfill
}
\makeatother



\author{Quentin Delhaye}
\title[Crypto using a Soc Platform]{Implementation of High-Level\\ Cryptographic Protocols using a SoC platform}
% \subtitle{}
\institute[ULB]{Université Libre de Bruxelles}
\date{June 24th, 2015}

\titlegraphic{\includegraphics[width=1.5cm]{ulbnorm}}
\titlegraphicii{\includegraphics[width=1.5cm]{logo-polytech-seul}}




%%%%%%%%%%%%%%%%%%%%%%%%
% data
%%%%%%%%%%%%%%%%%%%%%%%%
\def\temperaturedata{temperaturesOslo.txt}
\tikzstyle{maxmark} = [mark=*,mark options={color=red,scale=15}]
\tikzstyle{minmark} = [mark=*,mark options={color=blue,scale=15}]


\begin{document}

\begin{frame}[plain, noframenumbering]
\titlepage
\end{frame}

\begin{frame}
	%\tableofcontents[hideallsubsections]
	\tableofcontents
\end{frame}

\begin{frame}




\begin{tikzpicture}

%%%%%%%%%%%%%%%%%%%%%%%%
% throughput
%%%%%%%%%%%%%%%%%%%%%%%%
\begin{axis}[
        title = {FTP transfer inside an IPSec tunnel},
        width  = 0.5*\textwidth,
        height = 0.75\paperheight,
        major x tick style = transparent,
        xbar=10pt,
        bar width=8pt,
        % enlarge x limits={abs=1},
        % ymajorgrids = true,
        xlabel = {Throughput [MB/s]},
        ylabel = {},
        xmin=0, xmax=12,
        symbolic y coords={oot, none:none, aes256cbc:none, aes256cbc:sha256, aes256gcm},
        ytick = data,
        scaled x ticks = false,%Disable the *10^4 exponent applied to all y axis markings.
        legend style={at={(0.5,-0.15)}, anchor=north,legend columns=4},
        % enlarge x limits=0.1,
    ]

\addplot[style={black,fill=ForestGreen,mark=none}]
    coordinates {
        (0,none:none)
        (8.83,aes256cbc:none)
        (6.47,aes256cbc:sha256)
        (5.09,aes256gcm)
    };
    \label{software}

\addplot[style={black,fill=BrickRed,mark=none}]
    coordinates {
        (0,none:none)
        (8.52,aes256cbc:none)
        (5.80,aes256cbc:sha256)
        (0,aes256gcm)
    };
    \label{ba411e}

\addplot[style={black,fill=black,mark=none}]
    coordinates {
        (11.39,none:none)
        (0,aes256cbc:none)
        (0,aes256cbc:sha256)
        (0,aes256gcm)
    };
    \label{out-of-tunnel}%"tp" for "throughput"

\addplot[style={black,fill=MidnightBlue,mark=none}]
    coordinates {
        (10.21,none:none)
        (0,aes256cbc:none)
        (0,aes256cbc:sha256)
        (0,aes256gcm)
    };
    \label{inside tunnel}
% \legend{software, ba411e, out-of-tunnel, inside tunnel}
\end{axis}

\end{tikzpicture}
\end{frame}


\begin{frame}
\begin{tikzpicture}[xscale=0.01, yscale=0.01]
  % Draw the zero line
  \draw[ultra thin, black!50] (1,0)--(365,0);
  % Draw average temperature
  \draw[ultra thin, black!20] (1,7.5)--(365,5.5);
  % Plot temperature data and mark the maximum temperature
  \draw[ultra thin] plot[smooth,maxmark, mark indices={192} ] 
    file {\temperaturedata};
  % Draw the minimum temperature mark.     
  \draw[ultra thin] plot[smooth,only marks, minmark, mark indices={61}] 
    file {\temperaturedata};
\end{tikzpicture}
\end{frame}


% %%%%%%%%%%%%%%%%%%%%%%%%%%%%%%%%%%%%%%%%%%%%
% \section*{References}
% %%%%%%%%%%%%%%%%%%%%%%%%%%%%%%%%%%%%%%%%%%%%

% \nocite*{}
% \bibliographystyle{plain}
% \bibliography{bibliography}
		

\end{document}
