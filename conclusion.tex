\chapter{Discussion and conclusion}

\section{Future work}\label{sec:future-work}

Although the present work presents some promissing results, the implementation can certainly be improved in several ways and some further experiments should be conducted.

\paragraph{Driver improvement}
The driver of the BA411E can be made less ressources hungry by improving the initialisation of the descriptors and their linking, but the gain would not be significant enough to justify the time investment at this point.
% TODO refer to a stack analysis of an ipsec file transfer, showing the most expensive instructions. If I remember correctly, it was something like the DMA transfer and DMA something freeing.
A better alternative would be to avoid descriptors altogether by modifiying the interface with the IP so that it can use the scatterlist directly.
We would then spare a lot of DMA mapping instruction and thus some precious cycles on the software side.

As we already remarked in \ref{sec:section-where-we-see-the-CPU-usage-of-ba411e-usage-with-active-polling}, the use cases involving IPsec were conducted using a previous revision of the driver still activelly polling the IP for its results.
A better and cleaner way to proceed is to use interruption routines, as shown in \ref{sec:section-ba411e-with-irq}.
However, the kernel does not support their current implementation and panics upon usage.
If one were to be willing to spend the time replacing the active polling by clean asynchronous interruptions, he should be aware of the overhead imposed by an interruption.
In some cases, when the operation is just a few clock cycle long for the IP, an active polling could still be the better way to go.
A more thorough comparison of the mutual trade-off deserves some investigation, and as a starting point, the packet size could be treated as a branching point between the two solutions.


\paragraph{Registering public key verification with the crypto API}
As we saw in \ref{sec:section-with-RSA-BA414}, the driver is already capable of offloading a large portion of public key operations to the IP, but only with very specific libraries at the time being -- openSSL in our case.
The next step is to register the very same operations with the crypto API so it can be used without having to rely on a custom openSSL engine.
This feature would require to work very closely to the linux kernel developpement.
Indeed, if the signature verification using public key cryptography has been available in the kernel since 2013, a public key encryption API has only been proposed in late April 2015~\cite{crypto-api-pk-encryption} and is still under request for comments.

\paragraph{Conditional offloading in cryptodev}
%Note this is still under heavy criticism. If the hardware lags so far behind, why the relatively good score in openvpn?
The figure \ref{fig:openssl-benchmark-cbc} clearly shows a threshold on the packet size from which the hardware has a clear advantage, and below which the user mode software implementation is to go for.
%TODO include a listing to where it could be a good place to branch on the packet size. I already found it.
%TODO at least add a pseudo-code to illustrate this shit.
Using this breakeven point, one could patch the cryptodev cryptodev engine of OpenSSL to branch on the packet size, using the hardware if the packet is large enough, or fallback on the default software implementation otherwise. %openssl-1.0.2a/crypto/engine/eng_cryptodev.c
It would probably not be as trivial and beneficial as it may sound:
\begin{itemize}
	\item the encryption contexts would need to be synchronized between the hardware and the software;
	% \item we would already suffer from some context switches when sending the information to the cryptodev driver; %No, we are still in the openssl engine.
	% \item if we always branch to the fallback functions, the time overhead due to crytodev would lower the performance; %No, we don't leave the user space, the overhead is minimum.
	\item as the breakeven point is around 1024kB, the performance for a network application would be very close to those of a full software implementation, knowing that the ethernet frame size, the MTU, is set by default at 1500 bytes.
\end{itemize}

Such a conditional offloading would be interresting for applications involving mainly very large packets and a few periodic smaller ones, like large data transfer between two hosts on a infrastructure supporting ethernet jumbo frames\footnote{Jumbo frames have an ethernet MTU of 9000 bytes, whilst standard frames are set to 1500.} with periodic ICMP heartbeat.

\paragraph{Disk encryption}
As the hardware is better used with larger blocks of data, disk encryption could be an interesting application to look into.

\paragraph{Cryptographic libraries}
OpenSSL is not the only cryptographic library available; GnuTLS is also a very popular alternative and supports cryptodev engines too.

However, one library definitely worth to keep an eye on is mbed TLS, formally known as PolarSSL, recently bought by ARM~\cite{2015-arm-buy-polarssl}.
We can expect the future releases of this library to be more optimized for ARM platforms, and maybe the software footprint and overhead to be reduced.


\paragraph{Cryptodev}
If patches adding the GCM support to cryptodev have already been released, those are not compatible with Barco Silex' driver.
Adapting the interface would open the GCM hardware offload to the whole user space applications park.

\paragraph{MAC offloading}
While the symetric and asymetric encryption ciphers IPs are usable from the operating system, the IP computing MACs does not have a usable driver yet.
Wherever there is encryption, authentication is also needed.
As such, any real day-to-day use case can not be fully offloaded to hardware yet, even if some tricks and patches allowed us to bypass this requirement.
The implementation of the GCM mode, combining encryption and authentication, showed us that stopping relying on the software implementation of MACs would be a huge step forward.


% This work is just the beginning of something far larger. The aim is to build a fully secured platform, from software to hardware, using maximum security at each stage.