\chapter{Results and Analysis}\label{chap:results}

%%%%%%%%%%%%%%%%%%%%%%%%%%%%%%%%%%%%%%%%%%%%%%%%%%%%%%%%%%%%%%%%%%
%%%%%%%%%%%%%%%%%%%%%%%%%%%%%%%%%%%%%%%%%%%%%%%%%%%%%%%%%%%%%%%%%%
\section{TLS connections}
%%%%%%%%%%%%%%%%%%%%%%%%%%%%%%%%%%%%%%%%%%%%%%%%%%%%%%%%%%%%%%%%%%
%%%%%%%%%%%%%%%%%%%%%%%%%%%%%%%%%%%%%%%%%%%%%%%%%%%%%%%%%%%%%%%%%%
% Begin with the benchmark done by Bastien on the raw number of verif/s with openssl.

If the ten clients could connect instantaneously to the server every second, the maximum number of connections would be 600 per minute.
However, a certain connection time has to be taken into account.
Those are summarized in table~\ref{tab:openvpn-con-time}.

\begin{table}[ht]
\center
\small
\begin{tabular}{ll|l|} \cline{3-3}
 & & Connection time [s] \\ \hline
\multicolumn{1}{|l|}{\multirow{2}{*}{RSA-1024}} & soft & 0.041921 \\ \cline{2-3}
\multicolumn{1}{|l|}{} & BA411E & 0.020312 \\ \hline
\multicolumn{1}{|l|}{\multirow{2}{*}{RSA-2048}} & soft & 0.202945 \\ \cline{2-3}
\multicolumn{1}{|l|}{} & BA411E & 0.039965 \\ \hline
\multicolumn{1}{|l|}{\multirow{2}{*}{RSA-4096}} & soft & 1.436743 \\ \cline{2-3}
\multicolumn{1}{|l|}{} & BA411E & 0.183533 \\ \hline
\end{tabular}
\caption{OpenVPN connection time}{time necessary to establish an aes-256-cbc connection with DHE.}
\label{tab:openvpn-con-time}
\end{table}

It already shows that the connection latency is divided by 2 for low security RSA, and up to by 7.8 for higher security parameters.
The figure~\ref{fig:openvpn-tls-bench} shows the number of TLS connections per minute for three RSA exponent sizes: 1024-, 2048- and 4096-bit.
The higher the exponent size, the higher the performance boost.


\begin{table}[ht]
\center
\small
\begin{tabular}{l|c|c|c|c|c|c|c|c|c|} \cline{2-10}
 & \multicolumn{3}{c|}{RSA-1024} & \multicolumn{3}{c|}{RSA-2048} & \multicolumn{3}{c|}{RSA-4096} \\ \cline{2-10}
 & \multicolumn{2}{c|}{Con.} & CPU & \multicolumn{2}{c|}{Con.} & CPU & \multicolumn{2}{|c|}{Con.} & CPU \\ \hline
\multicolumn{1}{|c|}{Soft} & 445.4 & \multirow{2}{*}{x1.14} & 40.32 & 155.6& \multirow{2}{*}{x2.70}  & 92.14 & 19.6& \multirow{2}{*}{x5.89}  & 81.97 \\ \cline{1-2}\cline{4-5}\cline{7-8}\cline{10-10}
\multicolumn{1}{|c|}{BA414E} & 509.3 & & 13.29 & 420.9 & & 4.82 & 115.5 & & 4.34 \\ \hline
\end{tabular}
\caption{TLS connections per minute}{measures obtained with ten clients concurently connecting to an OpenVPN server.}
\label{tab:tls-con}
\end{table}

\begin{figure}[ht]
\center
\begin{tikzpicture}

%%%%%%%%%%%%%%%%%%%%%%%%
% CPU in the background
%%%%%%%%%%%%%%%%%%%%%%%%
\begin{axis}[
        width  = 0.7*\textwidth,
        height = 8cm,
        major x tick style = transparent,
        ybar=2pt,%space between the bars
        bar width=16pt,
        enlarge x limits={abs=1},
        ylabel = {CPU\#0 usage},
        hide x axis,
        axis y line*=right,
        ymin=0, ymax=100,
        symbolic x coords={RSA-1024, RSA-2048, RSA-4096},
        xtick = data,
        scaled y ticks = false,%Disable the *10^4 exponent applied to all y axis markings.
        legend style={at={(0.5,-0.15)}, anchor=north,legend columns=2},
        enlarge x limits=0.1,
    ]

\addplot[style={black,fill=LimeGreen,postaction={pattern=north east lines},mark=none}]
    coordinates {
        (RSA-1024, 40.32)
        (RSA-2048, 92.14)
        (RSA-4096, 81.97)
    };
    \label{software}

\addplot[style={black,fill=RedOrange,postaction={pattern=north east lines},mark=none}]
    coordinates {
        (RSA-1024, 13.29)
        (RSA-2048, 4.82)
        (RSA-4096, 4.78)
    };
    \label{ba414e}
\legend{software, ba414e}
\end{axis}

%%%%%%%%%%%%%%%%%%%%%%%%
% throughput
%%%%%%%%%%%%%%%%%%%%%%%%
\begin{axis}[
        title = {TLS connections through OpenVPN},
        width  = 0.7*\textwidth,
        height = 8cm,
        major x tick style = transparent,
        ybar=10pt,
        bar width=8pt,
        enlarge x limits={abs=1},
        ymajorgrids = true,
        ylabel = {Throughtput [MB/s]},
        xlabel = {},
        ymin=0,
        symbolic x coords={RSA-1024, RSA-2048, RSA-4096},
        xtick = data,
        scaled y ticks = false,%Disable the *10^4 exponent applied to all y axis markings.
        legend style={at={(0.5,-0.25)}, anchor=north,legend columns=2},
        enlarge x limits=0.1,
    ]

\addplot[style={black,fill=ForestGreen,mark=none}]
    coordinates {
        (RSA-1024, 445.4)
        (RSA-2048, 155.6)
        (RSA-4096, 19.6)
    };
    \label{soft-tp}

\addplot[style={black,fill=BrickRed,mark=none}]
    coordinates {
        (RSA-1024, 509.3)
        (RSA-2048, 420.9)
        (RSA-4096, 115.5)
    };
    \label{ba411e-tp}
\legend{}
\end{axis}

\end{tikzpicture}


\caption{TLS connections per minute}{The background stripped bars are the CPU usage. Raw data in table~\ref{tab:tls-con}.}
\label{fig:openvpn-tls-bench}
\end{figure}


\noindent For RSA-1024, the results are mitigated: a poor performance increase, but already less than half the CPU usage.
It should be noted that at this point, the number of clients is probably too low to push the configuration to its limits.
It is however an interesting comparison case with the next level of security: RSA-2048.

\noindent RSA-2048 is a much more common configuration, espacially since the NIST deprecated RSA-1024 in 2013.
The full software implementation is visibly affected by the increase of the exponent size: the CPU usage doubles and the server processes three time less connections.
At the same time, the hardware loses less than 20\% connections for a third of the CPU usage.
%TODO check if it is hardware limited or limited by the performances of the VM. It is not to be forgotten that it has to keep up with the server, even when the latter is helped by the high-performance hardware. My guess is that the VM is capping the perf, hence the exact same CPU usage for RSA-2048 and RSA-4096.

\noindent The results obtained for RSA-4096 can be interpreted similarly to those of RSA-2048, except that the CPU usage is exactly the same for th hardware configuration.
One way too look at those results is to directly compare the raw performance, and the hardware can then process almost six times more connections per minute than the software.
However, this is only half of it, since it leaves the CPU usage drop aside.
If we look at the efficiency, the software can process 0.24 connection per percentage of CPU usage, whilst the hardware can process 24 of them.
The efficiency is thus multipled by a factor 1000.

Such interesting results, particularly regarding the CPU usage, are possible because at least 87\% of the operations are RSA and Diffie-Hellman operations, which are entirely offloaded in hardware.
Nevertheless, OpenVPN still needs to proceed to some extra computations (such as SHA-1 integrity), and the hardware operations are not instantaneous, so the performance gain can only be that high.












%%%%%%%%%%%%%%%%%%%%%%%%%%%%%%%%%%%%%%%%%%%%%%%%%%%%%%%%%%%%%%%%%%
%%%%%%%%%%%%%%%%%%%%%%%%%%%%%%%%%%%%%%%%%%%%%%%%%%%%%%%%%%%%%%%%%%
\section{Response time -- latency}
%%%%%%%%%%%%%%%%%%%%%%%%%%%%%%%%%%%%%%%%%%%%%%%%%%%%%%%%%%%%%%%%%%
%%%%%%%%%%%%%%%%%%%%%%%%%%%%%%%%%%%%%%%%%%%%%%%%%%%%%%%%%%%%%%%%%%
The following tests are conducted after the connection has been established, so as the clients do not need to undergo any new key negociation.

\subsection{OpenVPN}
The figure~\ref{fig:ping-benchmark-openvpn} shows the results for different payload sizes.


\begin{figure}[ht]
%%%%%%%%%%%%%%%%%%%%%%%%%%%%%%%%%%%%%%%%%%%%%%%%%%%%%%%%%%
% Ping GCM
%%%%%%%%%%%%%%%%%%%%%%%%%%%%%%%%%%%%%%%%%%%%%%%%%%%%%%%%%%
\begin{tikzpicture}
\begin{axis}[
        title = {Ping benchmark -- OpenVPN},
        width  = \textwidth,
        height = 8cm,
        major x tick style = transparent,
        ybar,
        bar width=8pt,
        ymajorgrids = true,
        ylabel = {Response time [ms]},
        xlabel = {ICMP packet size [B]},
        ymin=0, ymax=10,
        symbolic x coords={56, 1000, 8000, 16000},
        xtick = data,
        scaled y ticks = false,%Disable the *10^4 exponent applied to all y axis markings.
        legend style={at={(0.5,-0.25)}, anchor=north,legend columns=3},
        enlarge x limits=0.2,
    ]
% I would have prefered to have "\addplot[marks only]", but they overlap if they have the same x coordinate,
% not like bars that automatically shift.
\addplot[style={black, fill=black}]
    coordinates {
        (56, 1.444)
        (1000, 1.929)
        (8000, 2.811)
        (16000, 4.322)
    };
    \label{raw}

\addplot[style={NavyBlue, fill=NavyBlue}]
    coordinates {
        (56, 2.066)
        (1000, 2.561)
        (8000, 4.293)
        (16000, 6.117)
    };
    \label{none-none}

\addplot[style={OliveGreen, fill=OliveGreen},postaction={pattern=north east lines}]
    coordinates {
        (56, 2.044)
        (1000, 2.760)
        (8000, 5.092)
        (16000, 7.166)
    };
    \label{soft-cbc256-none}

\addplot[style={BrickRed, fill=BrickRed},postaction={pattern=north east lines}]
    coordinates {
        (56, 2.301)
        (1000, 2.783)
        (8000, 5.377)
        (16000, 8.099)
    };
    \label{ba411e-cbc256-none}

\addplot[style={OliveGreen, fill=OliveGreen}]
    coordinates {
        (56, 2.415)
        (1000, 3.061)
        (8000, 5.997)
        (16000, 9.135)
    };
    \label{soft-cbc256-sha256}

\addplot[style={BrickRed, fill=BrickRed}]
    coordinates {
        (56, 2.416)
        (1000, 3.052)
        (8000, 5.963)
        (16000, 9.207)
    };
    \label{ba411e-cbc256-sha256}

\legend{raw, none-none, soft-cbc256-none, ba411e-cbc256-none, soft-cbc256-sha256, ba411e-cbc256-sha256}
\end{axis}
\end{tikzpicture}
% Here, I could show the gcm128, which show better performances with the BA411e, but I would be weird to compare it with aes256cbc.
% I need another graph with a CPu usage comparison to show that even if the perf are the same for soft/hard with aes256gcm, the hard loads less the CPU (I hope so, at least).
\caption{OpenVN: ping average response time}{}
\label{fig:ping-benchmark-openvpn}
\end{figure}

\subsection{IPsec}

\begin{figure}[ht]
%%%%%%%%%%%%%%%%%%%%%%%%%%%%%%%%%%%%%%%%%%%%%%%%%%%%%%%%%%
% Ping GCM
%%%%%%%%%%%%%%%%%%%%%%%%%%%%%%%%%%%%%%%%%%%%%%%%%%%%%%%%%%
\begin{tikzpicture}
\begin{axis}[
        title = {Ping benchmark -- IPsec},
        width  = \textwidth,
        height = 8cm,
        major x tick style = transparent,
        ybar,
        bar width=8pt,
        ymajorgrids = true,
        ylabel = {Response time [ms]},
        xlabel = {ICMP packet size [B]},
        ymin=0, ymax=10,
        symbolic x coords={56, 1000, 8000, 16000},
        xtick = data,
        scaled y ticks = false,%Disable the *10^4 exponent applied to all y axis markings.
        legend style={at={(0.5,-0.25)}, anchor=north,legend columns=3},
        enlarge x limits=0.15,
    ]
% I would have prefered to have "\addplot[marks only]", but they overlap if they have the same x coordinate,
% not like bars that automatically shift.
\addplot[style={black, fill=black}]
    coordinates {
        (56, 1.444)
        (1000, 1.929)
        (8000, 2.811)
        (16000, 4.322)
    };
    \label{raw}

\addplot[style={NavyBlue, fill=NavyBlue}]
    coordinates {
        (56, 1.545)
        (1000, 2.045)
        (8000, 2.997)
        (16000, 4.703)
    };
    \label{none-none}

\addplot[style={OliveGreen, fill=OliveGreen},postaction={pattern=north east lines}]
    coordinates {
        (56, 1.581)
        (1000, 2.134)
        (8000, 3.910)
        (16000, 6.426)

    };
    \label{soft-cbc256-none}

\addplot[style={BrickRed, fill=BrickRed},postaction={pattern=north east lines}]
    coordinates {
        (56, 1.639)
        (1000, 2.080)
        (8000, 3.445)
        (16000, 5.345)
    };
    \label{ba411e-cbc256-none}

\addplot[style={OliveGreen, fill=OliveGreen}]
    coordinates {
        (56, 1.645)
        (1000, 2.322)
        (8000, 4.762)
        (16000, 7.975)

    };
    \label{soft-cbc256-sha256}

\addplot[style={BrickRed, fill=BrickRed}]
    coordinates {
        (56, 1.635)
        (1000, 2.246)
        (8000, 4.170)
        (16000, 6.929)
    };
    \label{ba411e-cbc256-sha256}

\addplot[style={OliveGreen, fill=OliveGreen},postaction={pattern=north west lines}]
    coordinates {
        (56, 1.651)
        (1000, 2.388)
        (8000, 5.383)
        (16000, 9.241)
    };
    \label{soft-gcm256}

\addplot[style={BrickRed, fill=BrickRed},postaction={pattern=north west lines}]
    coordinates {
        (56, 1.764)
        (1000, 2.286)
        (8000, 0)
        (16000, 0)
    };
    \label{ba411e-gcm256}

\legend{raw, none-none, soft-cbc256-none, ba411e-cbc256-none, soft-cbc256-sha256, ba411e-cbc256-sha256, soft-gcm256, ba411e-gcm256}
\end{axis}
\end{tikzpicture}
% Here, I could show the gcm256, which show better performances with the BA411e, but I would be weird to compare it with aes256cbc.
% I need another graph with a CPu usage comparison to show that even if the perf are the same for soft/hard with aes256gcm, the hard loads less the CPU (I hope so, at least).
\caption{Ping min/avg/max response time}{for different packet sizes using IPsec. For each packet size, 1000 requests were flooded to the board, that is \textit{"outputs packets as fast as they come back or one hundred times per second, whichever is more"}, according to the \texttt{ping} command manual.} %TODO move the end of the caption to the "test protocol" chapter.
\label{fig:ping-benchmark-ipsec}
\end{figure}



\subsection{Comparison}

\begin{figure}[ht]

\begin{tikzpicture}
\begin{axis}[
        title = {Ping benchmark},
        width  = \textwidth,
        height = 8cm,
        major x tick style = transparent,
        ybar,
        bar width=8pt,
        ymajorgrids = true,
        ylabel = {Response time [ms]},
        xlabel = {ICMP packet size [B]},
        ymin=0,
        symbolic x coords={56, 1000, 4000, 8000, 16000},
        xtick = data,
        scaled y ticks = false,%Disable the *10^4 exponent applied to all y axis markings.
        legend style={at={(0.5,-0.25)}, anchor=north,legend columns=2},
        % enlarge x limits=0.1,
    ]

\addplot[style={brown, fill=brown, postaction={pattern=north east lines}}]
    coordinates {
        (56, 1.545)
        (1000, 2.045)
        (4000, 2.521)
        (8000, 2.997)
        (16000, 4.703)
    };
    \label{ipsec-none-none}

\addplot[style={brown, fill=brown, postaction={pattern=north west lines}}]
    coordinates {
        (56, 2.066)
        (1000, 2.561)
        (4000, 3.373)
        (8000, 4.293)
        (16000, 6.117)
    };
    \label{openvpn-none-none}

\addplot[style={green, fill=green, postaction={pattern=north east lines}}]
    coordinates {
        (56, 1.645)
        (1000, 2.322)
        (4000, 3.420)
        (8000, 4.762)
        (16000, 7.975)

    };
    \label{ipsec-soft-cbc256-sha256}

\addplot[style={green, fill=green, postaction={pattern=north west lines}}]
    coordinates {
        (56, 2.415)
        (1000, 3.061)
        (4000, 4.376)
        (8000, 5.997)
        (16000, 9.135)
    };
    \label{openvpn-soft-cbc256-sha256}

\addplot[style={orange, fill=orange, postaction={pattern=north east lines}}]
    coordinates {
        (56, 1.635)
        (1000, 2.246)
        (4000, 3.172)
        (8000, 4.170)
        (16000, 6.929)
    };
    \label{ipsec-ba411e-cbc256-sha256}

\addplot[style={orange, fill=orange, postaction={pattern=north west lines}}]
    coordinates {
        (56, 2.416)
        (1000, 3.052)
        (4000, 4.140)
        (8000, 5.963)
        (16000, 9.207)
    };
    \label{openvpn-ba411e-cbc256-sha256}

\legend{ipsec-none-none, openvpn-none-none, ipsec-soft-cbc256-sha256, openvpn-soft-cbc256-sha256, ipsec-ba411e-cbc256-sha256, openvpn-ba411e-cbc256-sha256}
\end{axis}
\end{tikzpicture}
% Here, I could show the gcm256, which show better performances with the BA411e, but I would be weird to compare it with aes256cbc.
% I need another graph with a CPu usage comparison to show that even if the perf are the same for soft/hard with aes256gcm, the hard loads less the CPU (I hope so, at least).
\caption{Ping average comparison}{All values are for aes-256-cbc with sha-256}
\label{fig:ping-benchmark-comparison}
\end{figure}
















%%%%%%%%%%%%%%%%%%%%%%%%%%%%%%%%%%%%%%%%%%%%%%%%%%%%%%%%%%%%%%%%%%
%%%%%%%%%%%%%%%%%%%%%%%%%%%%%%%%%%%%%%%%%%%%%%%%%%%%%%%%%%%%%%%%%%
\section{File transfer}
%%%%%%%%%%%%%%%%%%%%%%%%%%%%%%%%%%%%%%%%%%%%%%%%%%%%%%%%%%%%%%%%%%
%%%%%%%%%%%%%%%%%%%%%%%%%%%%%%%%%%%%%%%%%%%%%%%%%%%%%%%%%%%%%%%%%%
%FTP over openvpn and IPsec

%OpenSSH: 	- normal transfer: shows perf difference
%			- capped to software max: show CPU offload

\subsection{openSSH}

\begin{figure}[ht]
\begin{tikzpicture}

%%%%%%%%%%%%%%%%%%%%%%%%
% CPU in the background
%%%%%%%%%%%%%%%%%%%%%%%%
\begin{axis}[
        width  = 0.95*\textwidth,
        height = 8cm,
        major x tick style = transparent,
        ybar=2pt,%space between the bars
        bar width=16pt,
        enlarge x limits={abs=1},
        ylabel = {CPU\#0 usage},
        hide x axis,
        axis y line*=right,
        ymin=0, ymax=100,
        symbolic x coords={none:none, aes256cbc:none, aes256cbc:sha256},
        xtick = data,
        scaled y ticks = false,%Disable the *10^4 exponent applied to all y axis markings.
        legend style={at={(0.5,-0.15)}, anchor=north,legend columns=4},
        enlarge x limits=0.1,
    ]

\addplot[style={black,fill=LimeGreen,postaction={pattern=north east lines},mark=none}]
    coordinates {
        (none:none, 0)
        (aes256cbc:none, 96.26)
        (aes256cbc:sha256, 89.29)
    };
    \label{software}

\addplot[style={black,fill=RedOrange,postaction={pattern=north east lines},mark=none}]
    coordinates {
        (none:none, 0)
        (aes256cbc:none, 46.01)
        (aes256cbc:sha256, 68.19)
    };
    \label{ba411e}

\addplot[style={black,fill=gray,postaction={pattern=north east lines},mark=none}]
    coordinates {
        (none:none, 7.16)
        (aes256cbc:none, 0)
        (aes256cbc:sha256, 0)
    };
    \label{out-of-tunnel}%"oot" for "out of tunnel"
\legend{software, ba411e, out-of-tunnel}
\end{axis}

%%%%%%%%%%%%%%%%%%%%%%%%
% throughput
%%%%%%%%%%%%%%%%%%%%%%%%
\begin{axis}[
        title = {file transfer over SSH tunnel},
        width  = 0.95*\textwidth,
        height = 8cm,
        major x tick style = transparent,
        ybar=10pt,
        bar width=8pt,
        enlarge x limits={abs=1},
        ymajorgrids = true,
        ylabel = {Throughtput [MB/s]},
        xlabel = {},
        ymin=0, ymax=12,
        symbolic x coords={oot, none:none, aes256cbc:none, aes256cbc:sha256},
        xtick = data,
        scaled y ticks = false,%Disable the *10^4 exponent applied to all y axis markings.
        legend style={at={(0.5,-0.25)}, anchor=north,legend columns=2},
        enlarge x limits=0.1,
    ]

\addplot[style={black,fill=ForestGreen,mark=none}]
    coordinates {
        (none:none, 0)
        (aes256cbc:none, 10.89)
        (aes256cbc:sha256, 8.19)
    };
    \label{soft-tp}

\addplot[style={black,fill=BrickRed,mark=none}]
    coordinates {
        (none:none, 0)
        (aes256cbc:none, 10.67)
        (aes256cbc:sha256, 10.39)
    };
    \label{ba411e-tp}

\addplot[style={black,fill=black,mark=none}]
    coordinates {
        (none:none, 11.39)
        (aes256cbc:none, 0)
        (aes256cbc:sha256, 0)
    };
    \label{oot-tp}%"tp" for "throughput"
\legend{}
\end{axis}

\end{tikzpicture}
\caption{file transfer over an SSH tunnel. The background stripped bars are the CPU usage.}{}
\label{fig:openssh-bench}
\end{figure}

\subsection{OpenVPN}

\begin{figure}[ht]
\begin{tikzpicture}

%%%%%%%%%%%%%%%%%%%%%%%%
% CPU in the background
%%%%%%%%%%%%%%%%%%%%%%%%
\begin{axis}[
        width  = 0.95*\textwidth,
        height = 8cm,
        major x tick style = transparent,
        ybar=2pt,%space between the bars
        bar width=16pt,
        enlarge x limits={abs=1},
        ylabel = {CPU\#0 usage},
        hide x axis,
        axis y line*=right,
        ymin=0, ymax=100,
        symbolic x coords={none:none, aes256cbc:none, aes256cbc:sha256},
        xtick = data,
        scaled y ticks = false,%Disable the *10^4 exponent applied to all y axis markings.
        legend style={at={(0.5,-0.15)}, anchor=north,legend columns=4},
        enlarge x limits=0.1,
    ]

\addplot[style={black,fill=LimeGreen,postaction={pattern=north east lines},mark=none}]
    coordinates {
        (none:none, 0)
        (aes256cbc:none, 76.60)
        (aes256cbc:sha256, 76.03)
    };
    \label{software}

\addplot[style={black,fill=RedOrange,postaction={pattern=north east lines},mark=none}]
    coordinates {
        (none:none, 0)
        (aes256cbc:none, 83.74)
        (aes256cbc:sha256, 80.89)
    };
    \label{ba411e}

\addplot[style={black,fill=gray,postaction={pattern=north east lines},mark=none}]
    coordinates {
        (none:none, 7.16)
        (aes256cbc:none, 0)
        (aes256cbc:sha256, 0)
    };
    \label{out-of-tunnel}%"oot" for "out of tunnel"

\addplot[style={black,fill=brown,postaction={pattern=north east lines},mark=none}]
    coordinates {
        (none:none, 42.60)
        (aes256cbc:none, 0)
        (aes256cbc:sha256, 0)
    };
    \label{inside tunnel}%"it" for "in tunnel"
\legend{software, ba411e, out-of-tunnel, inside tunnel}
\end{axis}

%%%%%%%%%%%%%%%%%%%%%%%%
% throughput
%%%%%%%%%%%%%%%%%%%%%%%%
\begin{axis}[
        title = {FTP transfer inside OpenVPN tunnel},
        width  = 0.95*\textwidth,
        height = 8cm,
        major x tick style = transparent,
        ybar=10pt,
        bar width=8pt,
        enlarge x limits={abs=1},
        ymajorgrids = true,
        ylabel = {Throughtput [MB/s]},
        xlabel = {},
        ymin=0, ymax=12,
        symbolic x coords={oot, none:none, aes256cbc:none, aes256cbc:sha256},
        xtick = data,
        scaled y ticks = false,%Disable the *10^4 exponent applied to all y axis markings.
        legend style={at={(0.5,-0.25)}, anchor=north,legend columns=2},
        enlarge x limits=0.1,
    ]

\addplot[style={black,fill=ForestGreen,mark=none}]
    coordinates {
        (none:none, 0)
        (aes256cbc:none, 4.78)
        (aes256cbc:sha256, 3.87)
    };
    \label{soft-tp}

\addplot[style={black,fill=BrickRed,mark=none}]
    coordinates {
        (none:none, 0)
        (aes256cbc:none, 3.35)
        (aes256cbc:sha256, 2.84)
    };
    \label{ba411e-tp}

\addplot[style={black,fill=black,mark=none}]
    coordinates {
        (none:none, 11.39)
        (aes256cbc:none, 0)
        (aes256cbc:sha256, 0)
    };
    \label{oot-tp}%"tp" for "throughput"

\addplot[style={black,fill=RawSienna,mark=none}]
    coordinates {
        (none:none, 5.18)
        (aes256cbc:none, 0)
        (aes256cbc:sha256, 0)
    };
    \label{it-tp}
\legend{}
\end{axis}

\end{tikzpicture}
\caption{FTP file transfer over an OpenVPN tunnel. The background stripped bars are the CPU usage.}{}
\label{fig:openvpn-ftp-bench}
\end{figure}

We can see that adding a MAC computation aside the encryption merely lowers the performance when using the hardware.
Even though OpenSSL uses here an hihgly ARM-optimzed assembly implementation of SHA-256, it shows that the bottleneck is on the hardware side.

Indeed, 

\subsection{IPsec}

\begin{figure}[ht]
\begin{tikzpicture}

%%%%%%%%%%%%%%%%%%%%%%%%
% CPU in the background
%%%%%%%%%%%%%%%%%%%%%%%%
\begin{axis}[
        width  = 0.95*\textwidth,
        height = 8cm,
        major x tick style = transparent,
        ybar=2pt,%space between the bars
        bar width=16pt,
        enlarge x limits={abs=1},
        ylabel = {CPU\#0 usage},
        hide x axis,
        axis y line*=right,
        ymin=0, ymax=100,
        symbolic x coords={none:none, aes256cbc:none, aes256cbc:sha256, aes256gcm},
        xtick = data,
        scaled y ticks = false,%Disable the *10^4 exponent applied to all y axis markings.
        legend style={at={(0.5,-0.15)}, anchor=north,legend columns=4},
        enlarge x limits=0.1,
    ]

\addplot[style={black,fill=LimeGreen,postaction={pattern=north east lines},mark=none}]
    coordinates {
        (none:none, 0)
        (aes256cbc:none, 63.74)
        (aes256cbc:sha256, 74.64)
        (aes256gcm, 89.66)
    };
    \label{software}

\addplot[style={black,fill=RedOrange,postaction={pattern=north east lines},mark=none}]
    coordinates {
        (none:none, 0)
        (aes256cbc:none, 14.87)
        (aes256cbc:sha256, 17.25)
        (aes256gcm, 0)
    };
    \label{ba411e}

\addplot[style={black,fill=gray,postaction={pattern=north east lines},mark=none}]
    coordinates {
        (none:none, 7.16)
        (aes256cbc:none, 0)
        (aes256cbc:sha256, 0)
        (aes256gcm, 0)
    };
    \label{out-of-tunnel}%"oot" for "out of tunnel"

\addplot[style={black,fill=NavyBlue,postaction={pattern=north east lines},mark=none}]
    coordinates {
        (none:none, 14.68)
        (aes256cbc:none, 0)
        (aes256cbc:sha256, 0)
        (aes256gcm, 0)
    };
    \label{inside tunnel}%"it" for "in tunnel"
\legend{software, ba411e, out-of-tunnel, inside tunnel}
\end{axis}

%%%%%%%%%%%%%%%%%%%%%%%%
% throughput
%%%%%%%%%%%%%%%%%%%%%%%%
\begin{axis}[
        title = {FTP transfer insed IPSec tunnel},
        width  = 0.95*\textwidth,
        height = 8cm,
        major x tick style = transparent,
        ybar=10pt,
        bar width=8pt,
        enlarge x limits={abs=1},
        ymajorgrids = true,
        ylabel = {Throughtput [MB/s]},
        xlabel = {},
        ymin=0, ymax=12,
        symbolic x coords={oot, none:none, aes256cbc:none, aes256cbc:sha256, aes256gcm},
        xtick = data,
        scaled y ticks = false,%Disable the *10^4 exponent applied to all y axis markings.
        legend style={at={(0.5,-0.25)}, anchor=north,legend columns=2},
        enlarge x limits=0.1,
    ]

\addplot[style={black,fill=ForestGreen,mark=none}]
    coordinates {
        (none:none, 0)
        (aes256cbc:none, 8.83)
        (aes256cbc:sha256, 6.47)
        (aes256gcm, 5.09)
    };
    \label{soft-tp}

\addplot[style={black,fill=BrickRed,mark=none}]
    coordinates {
        (none:none, 0)
        (aes256cbc:none, 8.52)
        (aes256cbc:sha256, 5.80)
        (aes256gcm, 0)
    };
    \label{ba411e-tp}

\addplot[style={black,fill=black,mark=none}]
    coordinates {
        (none:none, 11.39)
        (aes256cbc:none, 0)
        (aes256cbc:sha256, 0)
        (aes256gcm, 0)
    };
    \label{oot-tp}%"tp" for "throughput"

\addplot[style={black,fill=MidnightBlue,mark=none}]
    coordinates {
        (none:none, 10.21)
        (aes256cbc:none, 0)
        (aes256cbc:sha256, 0)
        (aes256gcm, 0)
    };
    \label{it-tp}
\legend{}
\end{axis}

\end{tikzpicture}
\caption{FTP file transfer over an IPsec tunnel. The background stripped bars are the CPU usage.}{}
\label{fig:ipsec-ftp-bench}
\end{figure}

Some results have to be put into perspective with the fact that the implementation of SHA-256 is entirely C-based.
A more recent one using assembly instructions optimized for the NEON SIMD instruction set of the ARMv7 core could be used and would most probably yield better results.
The CPU usage of the software implementation would drop -- even if not significantly -- as for the hardware, it would be less limited by the software MAC counter part, and if the CPU usage could stay at the same level, we could expect a better throughput.

The GCM performance presented clearly shows a drop of throughput and an increase of CPU usage, illustrating the fact that those operations are hard on the software.
With an hardware offload, we could expect not only a drastic drop of the CPU usage, but an increase of throughput as well, since it's CPU-limited in those results.
Note that they are achieved usinf a C-based implementation of galois-field multiplications.
As we saw in chapter~\ref{chap:theory}, modern processor designers tend to add specialized instruction set aimed at AES-GCM enhancement.
Should further tests be conducted concerning IPsec paired with GCM, it would be wise to compare with an assembly implementation exploiting ARM NEON instruction set.
Some are being developed~\cite{Conrado2013,Danilo2013}, but none have been committed to the Linux kernel repository yet.

\subsection{Comparison}

\begin{figure}[ht]
\begin{tikzpicture}

%%%%%%%%%%%%%%%%%%%%%%%%
% CPU in the background
%%%%%%%%%%%%%%%%%%%%%%%%
\begin{axis}[
        width  = 0.95*\textwidth,
        height = 8cm,
        major x tick style = transparent,
        ybar=2pt,%space between the bars
        bar width=16pt,
        enlarge x limits={abs=1},
        ylabel = {CPU\#0 usage},
        hide x axis,
        axis y line*=right,
        ymin=0, ymax=100,
        symbolic x coords={Out-of-tunnel, openSSH, OpenVPN, IPSec},
        xtick = data,
        scaled y ticks = false,%Disable the *10^4 exponent applied to all y axis markings.
        legend style={at={(0.5,-0.15)}, anchor=north,legend columns=4},
        enlarge x limits=0.1,
    ]

\addplot[style={black,fill=LimeGreen,postaction={pattern=north east lines},mark=none}]
    coordinates {
        (Out-of-tunnel, 0)
        (openSSH, 89.29)
        (OpenVPN, 76.03)
        (IPSec, 74.64)
    };
    \label{software}

\addplot[style={black,fill=RedOrange,postaction={pattern=north east lines},mark=none}]
    coordinates {
        (Out-of-tunnel, 0)
        (openSSH, 68.29)
        (OpenVPN, 80.89)
        (IPSec, 17.25)
    };
    \label{ba411e}

\addplot[style={black,fill=gray,postaction={pattern=north east lines},mark=none}]
    coordinates {
        (Out-of-tunnel, 7.16)
        (openSSH, 0)
        (OpenVPN, 0)
        (IPSec, 0)
    };
    \label{out-of-tunnel}%"oot" for "out of tunnel"
\legend{software, ba411e, out-of-tunnel}
\end{axis}

%%%%%%%%%%%%%%%%%%%%%%%%
% throughput
%%%%%%%%%%%%%%%%%%%%%%%%
\begin{axis}[
        title = {Comparison over file transfer methods},
        width  = 0.95*\textwidth,
        height = 8cm,
        major x tick style = transparent,
        ybar=10pt,
        bar width=8pt,
        enlarge x limits={abs=1},
        ymajorgrids = true,
        ylabel = {Throughtput [MB/s]},
        xlabel = {},
        ymin=0, ymax=12,
        symbolic x coords={Out-of-tunnel, openSSH, OpenVPN, IPSec},
        xtick = data,
        scaled y ticks = false,%Disable the *10^4 exponent applied to all y axis markings.
        legend style={at={(0.5,-0.25)}, anchor=north,legend columns=2},
        enlarge x limits=0.1,
    ]

\addplot[style={black,fill=ForestGreen,mark=none}]
    coordinates {
        (Out-of-tunnel, 0)
        (openSSH, 8.19)
        (OpenVPN, 3.87)
        (IPSec, 6.47)
    };
    \label{soft-tp}

\addplot[style={black,fill=BrickRed,mark=none}]
    coordinates {
        (Out-of-tunnel, 0)
        (openSSH, 10.39)
        (OpenVPN, 2.84)
        (IPSec, 5.80)
    };
    \label{ba411e-tp}

\addplot[style={black,fill=black,mark=none}]
    coordinates {
        (Out-of-tunnel, 11.39)
        (openSSH, 0)
        (OpenVPN, 0)
        (IPSec, 0)
    };
    \label{oot-tp}%"tp" for "throughput"
\legend{}
\end{axis}

\end{tikzpicture}
\caption{Comparison of file transfer methods}{The background stripped bars are the CPU usage.}
\label{fig:ftp-bench-comparison}
\end{figure}