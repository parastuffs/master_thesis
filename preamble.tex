
\usepackage[utf8]{inputenc}
\usepackage[T1]{fontenc}
\usepackage{charter}
% \usepackage[bitstream-charter]{mathdesign}
\usepackage[francais,english]{babel}
\usepackage[x11names, rgb, usenames, dvipsnames]{xcolor}
\usepackage{geometry}
\usepackage[pdftex]{graphicx}
\usepackage{tikz}
\usepackage{pgfplots}%Needed at least by {axis} environment
\pgfplotsset{compat = 1.10}%1.3 needed by 'ylabel shift = 1 em'

\usepackage{graphicx}
\usepackage{listings}
\usepackage{picture}
\usepackage{epic}
\usepackage{verbatim}
\usepackage{textcomp}
\usepackage{tabularx}
\usepackage{multirow}
\usepackage{array}
% \usepackage[breaklinks=true,hidelinks]{hyperref}
\usepackage{memhfixc}
\usepackage[breaklinks=true]{hyperref}

\usepackage{amsmath}
\usepackage{dsfont} %Needed for \mathds{N}

\usepackage[numbers]{natbib}

\usepackage{framed}
\usepackage{listings}
\definecolor{colKeys}{rgb}{0,0,1}  % couleurs des mots-clés propres au language
\definecolor{colIdentifier}{rgb}{0,0,0}  % couleurs des mots à identifier
\definecolor{colComments}{rgb}{0,0.6,0}  % couleurs des commentaires
\definecolor{colString}{rgb}{0.6,0.1,0.1}

\setcounter{secnumdepth}{3}% Number up to subsubsection
\maxsecnumdepth{subsubsection}% 'secnumdepth' is reset by \mainmatter. You should use 'maxsecnumdepth'.
% \seccounter{tocdepth}{3}% Add subsubsection to the table of content.

% ----------------------
% lists
% ----------------------
\usepackage{enumitem}
% \setlist{nosep} % remove all vertical sep inside and around the list
\setlist{noitemsep} % remove all vertical spaces inside the list
% \setlist[2]{noitemsep} % set the itemsep and parsep for all level two lists to 0
% \setenumerate{noitemsep} % set noitemseponly for enumerate
% \begin{description}[noitemsep] % if the \setlist{noitemsep} as not been defined, use 'noitemsep' only for this list.
% \end{description}


% ----------------------
% Cover page
% ----------------------

% Logos
\newcommand{\ulb}{\includegraphics[scale=1.1]{logo_ULB2.pdf}}
\newcommand{\polytech}{\includegraphics[scale=0.35]{logo_polytech_FR.pdf}}

% Polices
\definecolor{ULBblue}{rgb}{0,0.2196,0.5765}
\newcommand{\fontTitle}{\sffamily \Huge\selectfont \color{ULBblue}}
\newcommand{\fontSubtitle}{\sffamily \LARGE \selectfont \color{ULBblue}}
\newcommand{\fontText}{\sffamily \selectfont}
\newcommand{\fontColor}{\sffamily \selectfont \color{ULBblue}}

% Titre
\newcommand{\titleA}{\fontTitle{Implementation of High-Level Cryptographic}} % Titre identique au titre remis au secrétariat
\newcommand{\titleB}{\fontTitle{Protocols using a SoC Platform}} % (dans la langue de rédaction a priori)
% Sous-titre
%\newcommand{\subtitle}{\fontSubtitle{Ligne du sous-titre du mémoire}}
% Titre du diplôme
\newcommand{\diplomaA}{\fontText{Mémoire présenté en vue de l’obtention du diplôme}} % A laisser en Français
\newcommand{\diplomaB}{\fontText{d'Ingénieur Civil en informatique à finalité spécialisée}}

% Etudiant
\newcommand{\student}{\textbf{\sffamily \large Quentin Delhaye}}

% Supervision
\newcommand{\promAa}{\fontColor{Directeur}}
\newcommand{\promAb}{\fontText{Professeur Frédéric Robert}}
% \newcommand{\promBa}{\fontColor{Co-Promoteur}}
% \newcommand{\promBb}{\fontText{Professeur [Prénom Nom]}}
\newcommand{\promCa}{\fontColor{Superviseur}}
\newcommand{\promCb}{\fontText{Sébastien Rabou (Barco Silex)}}
\newcommand{\deptA}{\fontColor{Service}}
\newcommand{\deptB}{\fontText{BEAMS}}

% Année académique
\newcommand{\yearA}{\fontColor{Année académique}}
\newcommand{\yearB}{\fontText{2014 - 2015}}